\chapter{Project Definition}
\label{chap:project_def}

\section{Document Information}

\begin{description}
\item[Project Name] RotaSense
\item[Client] Dr. Kate Enzler, physical therapist at Alexian Rehab Hospital (in
  partnership with Shirley Ryan AbilityLab)
\item[Team Members] Maddie, Andrew, Nikola, Kathryn
\item[Date] 02/14/2023
\item[Version] 3
\end{description}

\section{Project Information}

\subsection{Mission Statement}

Design a simple, lightweight, and adjustable device for daily use by
individuals experiencing left-neglect to draw focus to the user’s left side,
enhance user autonomy, and improve general quality of life. 

\subsection{Project Deliverables}

The end result of our project will be

\begin{itemize}
\item A final prototype.
\item A final report.
\item A final presentation in the form of a poster.
\end{itemize}

\subsection{Constraints}

We have two main constraints on this DTC project, namely time (10 weeks) and
budget (\$100).

\section{User Information}

\subsection{Users and Stakeholders}

The users and stakeholders for our project are
\begin{itemize}
\item People who are experiencing left side neglect.
\item Therapists who work with patients on mitigating the effects of left side
  neglect. 
\item Family members/caregivers who are responsible for reminding the patients
  to look left at home. 
\end{itemize}

\subsection{Illustrative User Scenario}

The user in the illustrative scenario below is based on an observation of a
patient in therapy for left-neglect and supported by details from an interview
with Dr. Kate Enzler.

The user begins their therapy session with Dr. Enzler at the Alexian Rehab
Hospital. The patient has very recently had a stroke, and their eyes tend to
look to the right side. When asked to perform the lighthouse scanning technique
(a technique prompts someone to turn their head completely from right to left
like a lighthouse), the user initially does not turn their head to the left,
stopping instead at the midline. The user only continues scanning to the left
when given a verbal cue by Dr. Enzler. There are people on the user’s left, but
the user only makes eye contact with the people in front and to the right of
them. Dr. Enzler then puts up a finger on the user’s left side, and the user is
insistent that there is no finger there, assuming that Dr. Enzler is tricking
them and believing that they have turned their head as far left as possible.

\section{Project Requirements}

\begin{table}[H]
  \centering
  \begin{tabular}{| c | c | c | m{10.5em}| m{10.5em} |}\hline
    Needs & Metrics & Units & Ideal Value & Allowable Value\\\hline\hline
    \multirow{2}{*}{Lightweight}
          & Weight & \unit{\g} & Less than \qty{25}{\g} & Less than
                                                          \qty{50}{\g}\\\cline{2-5} 
          & Max Size & \unit{\cm}
                            & Size of an average pair of glasses
                                          & Average width of a human body\\\hline
    \multirow{3}{*}{Durable}
          & Lifespan & Years & Around 2 year & Around 1 years\\\cline{2-5}
          & Battery Life & Months & More than 3 months & More than 1 month\\\cline{2-5}
          & Element-resistant & N/A & Yes & Yes\\\hline
    Affordable & Price & Dollars & Less than \$50 & Less than \$100\\\hline
  \end{tabular}
  
  \caption{Project requirements and metrics.}
\end{table}


%%% Local Variables:
%%% mode: latex
%%% TeX-master: "../final_report"
%%% End:
