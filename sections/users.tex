\chapter{Users and Requirements}
\label{chap:users}

\section{Main Users of the Design}

\subsection{Adults who are experiencing Left-Side Neglect}

The users for our design will be patients at the Alexian Brothers Rehabilitation
Center who have suffered a brain injury leaving them with inattentiveness to the
left side of their body and surroundings. They need cues and reminders from a
caretaker in order to remember to continue turning their head all the way to the
left past the midline. They would like a solution that is easy to use and that
reduces reliance on a caretaker in order to create more autonomy for the
patient.

\subsection{Physical therapists at Alexian Brothers Rehabilitation Center}

The physical therapists work closely with the users to help them recover from
their stroke and teach them techniques to mitigate the effects of left
neglect. 

\section{Requirements}

We identified the following requirements during the course of the quarter. The
design had to satisfy each of them. 

\subsection{Lightweight}

The device will be attached to a pair of glasses and therefore must be light
enough to comfortably fit on the glasses and not put too much weight on the
patient. It must be small enough to fit onto the glasses and not make them
lopsided.

\subsection{Durable}

Since the device is attached to the glasses that will be worn frequently by the
user, the device must be able to endure being worn daily as well as have a long
battery life. The device must have some protective covering so that the patient
is able to wear the glasses outside and so that the device can withstand being
dropped from a reasonable height.

\subsection{Customizable and Adjustable}

The degree of left neglect and the most effective stimuli can vary between
users. Thus, the device should facilitate the customizability of the stimuli
provided by the other components of the take home kit. 

%%% Local Variables:
%%% mode: latex
%%% TeX-master: "../final_report"
%%% End:
